
\documentclass[journal]{IEEEtran}
\usepackage{graphicx}


% *** GRAPHICS RELATED PACKAGES ***
%
\ifCLASSINFOpdf
\else
\fi



% correct bad hyphenation here
\hyphenation{op-tical net-works semi-conduc-tor}


\begin{document}
%
% paper title
% can use linebreaks \\ within to get better formatting as desired
\title{A Path Planner for Unmanned Surface Vehicles: A Survey}
%
%
% author names and IEEE memberships
% note positions of commas and nonbreaking spaces ( ~ ) LaTeX will not break
% a structure at a ~ so this keeps an author's name from being broken across
% two lines.
% use \thanks{} to gain access to the first footnote area
% a separate \thanks must be used for each paragraph as LaTeX2e's \thanks
% was not built to handle multiple paragraphs
%

\author{Riccardo Polvara% <-this % stops a space
\thanks{Riccardo Polvara is a PhD student with the Marine Science and Engineering School, University of Plymouth, Plymouth, England.
      E-mail: {\tt riccardo.polvara@plymouth.ac.uk}}}




% make the title area
\maketitle


\begin{abstract}
%\boldmath
Developing a robust obstacle avoidance module is a foundamental step towards fully autonomous USVs. Until now, most of them move in the sea following way points paths,
 usually GPS-based, totally unaware about possible collisions against rocks, other vessels or also divers. In this paper, the actual state of the art regarding obstacle avoidance
 and the plan of a safe path between a starting point and a goal is summarized.
\end{abstract}


% Note that keywords are not normally used for peerreview papers.
\begin{IEEEkeywords}
Path planner, USV, control, obstacle avoidance.
\end{IEEEkeywords}



% For peer review papers, you can put extra information on the cover
% page as needed:
% \ifCLASSOPTIONpeerreview
% \begin{center} \bfseries EDICS Category: 3-BBND \end{center}
% \fi
%
% For peerreview papers, this IEEEtran command inserts a page break and
% creates the second title. It will be ignored for other modes.
%\IEEEpeerreviewmaketitle



\section{Introduction} \label{introduction}
Marine robots represent one of the three big families in which mobile robotics could be divided, together with terrain and aerial robots. This kind of vehicles can be also distinguished in Unmanned Surface Vehicles (USVs) or Unmanned Underwater vehicles (UUVs) based on the fact they operate at the same level of the sea or under it.
The interest toward them and the advantages they can provide started in the last century, with the development of the COMOX torpedo concept by Canadians in 1944 as a pre-Normandy invasion USV designed to lay smoke during the invasion \cite{Bertram2008}.\\
\indent Sponsored primarly by the US Navy, multiple platforms were developed and deployed in the late 1990s with the scope of reconnaissance and surveillance missions. Between these, an example is given by the Owl MK II, a Jet Ski chassis equipped with a low-profile hull for increased stealth and payload capability, a sonar and a video camera. The military community has expressed strong interest in the use of USVs for a variety of roles, including force protection, surveillance, min warfare, anti-submarine warfarem riverine operations and special forces operations. This interest has been focused primarly on the development of the Spartan USV by the US Space and Naval Warfare System Center in San Diego since 2003. \\
\indent Multiple unmanned marine vehicles have been built also outside the USA: in Japan, for example, Yamaha developed the Unmanned Marine Vehicle High-Speed UMV-H and the Unmanned marine Vehicle Ocean type UMV-O, involved in bio-geo-chemical monitoring. Other examples are the Canadian Barracuda, the Dolphin MK II, the Seal USV and the SARPAL AMV, all developed by the International Submarine Engineering Ltd (ISE),the Stingray used by the Israeli navy, the Delfim and Caravela developed by the Portuguese Dynamical Systems and Ocean Robotics lab, and finally the Springer developed by the University of Plymouth.\\
\indent Most of the vessel cited before are dual-purpose vehicles, able to function in the conventional manned mode or in the unmanned one. Sometimes the installation of remoting kit allows the craft to be operated in fully manual mode, in autopilot-augmented mode or in remote-contrl mode. In this way the ship not only retains full manual capability but that capability is augmented and extended in an affordable and low-risk manner.\\
\indent To navigate in a fully autonomous way, marine vehicles require the presence of an \textit{obstacle avoidance module} able to move the vessel from the actual track to another one if an immediate collision is expected, and then take it back on the previous one towards the goal pose. As it usually happens with terrain robots, a path planner should be implemented: often it is distinguinshed in \textit{global path planner} (GPP) and \textit{local path planner} (LPP). The goal of GPP is to find a safe path connecting the starting position, the actual pose of the robot, and the final one, called \textit{goal pose}. In literature sometimes it is also called \textit{deliberative path planner}. Otherwise, the LPP has to react to immediate collision against obstacles unexpected or not considered by the GPP, moving the autonomous vehicles far from the preplanned path in order to avoid the moving obstacle; for this reason it is also called \textit{reactive path planner}.\\
\indent The structure of the paper is divided as follows: in Section \ref{obs_det} I illustrate how to perceive the environment surronding the autonomous vessel and detect static and moving obstacles with the most used computer vision techniques; in Sect. \ref{path_planner} I discuss more in details the necessity of having a robust path planner, therefore in subsections \ref{gpp} and \ref{lpp} I illustrate how a global and a local path planners, respectively, could be implemented. Finally, in Section \ref{conclusion} I propose a new combined path planner based on A* and other techniques presented in the previous sections.


\section{Obstacle Detection} \label{obs_det}
\indent To perfectly avoid the obstacles moving along the path of the autonomous vessel, an highly accurated world model is required. In order to obtain it, different sensors can be combined and data coming from them are usually fused in a 2D or 3D representation.\\
\indent In \cite{Almeida2009} the authors suggest to use an ARP radar sensor to identify moving obstacles and shores, and classify targets in terms of collision threat. They identify a set of perimeters around the USV in order to decide appropriate measures: \textit{irrilevant} perimeter(3km), \textit{safe} perimeter(500m), \textit{warning} perimeter(250m) and \textit{prohibition} one(50m). Based on the \textit{Closest Point of Approach} (CPA), defined as the estimated distance between the USV and the detected object at the time in which such distant is minimal, they classify targets as \textit{No Threat} (CPA outside Irrilevant perimeter), \textit{Low Threat} (CPA crosses the Irrilevenat perimeter but not the Safe one), \textit{Potential Threat} (CPA crosses the safe perimeter but not the Prohibited one) and \textit{Dangerous} (CPA inside the Prohibition perimeter).\\
\indent Low-Cost radars are also used in the work of Schuster \cite{Schuster2014}. The aim of this work is to propose an on-board collision avoidance approach for those situations in which the Automatic Identification System (AIS) is unavailable, and therefore the position, course, speed and dimensions of other vessels have to be estimated by radar measurements.\\
\indent In order to detect objects, data coming from radars need to be image preprocessing in the following way:
\begin{itemize}
\item Ego Motion Compensation: the azimuth of each scan in respect to the corresponging vessel's heading is calculated to compensate the vessel's yaw rate;
\item Occupancy Likelihood Determination: assuming that the probability \textit{p} of a cell being reported as occupied is independent and constant, the occupancy likelihood is binomially distributed;
\item Connected Component Labeling: a cell is assumed to contain a target if the occupancy probability is greater than 0.5; since a target usually extends of several hundred cells, adjacent, occupied cells are grouped using connected component labeling and each group is considered to be a target of elliptical shape.
\end{itemize}
The extracted target positions are very noisy, thus strong low pass filtering is required to obtain an object's true position, heading and velocity. To this scope, an \textit{Interactive Multiple Model} (IMM) filter is chosen: it runs several models in parallel and, based on the estimate of each model and the current measurement, a likelihood for each model to reflect the true motion state is determined. The output of the filter is a weighted sum of all model estimates and it is used by the collisions avoidance algorithm to predict the movement of the other vessels.
\indent Other approaches use monocular and stereo vision methods for recording the presence of obstacles in proximity (30 to 100 meters) of the vessel. An example is offered by the work of Wang \textit{et al.} \cite{Wang2011,Wang2012} in which two cameras are mounted parallel about 1.5 meters apart on a metal bar: the image from the camera on the left is initially used to perform monocular obstacle detection, and then the stero approach is applied to process the image from both cameras to compute the 3D detection results. The monocular technique is composed as follows:
\begin{itemize}
\item Horizon detection module: it allows to distinguish the sea surface for obstacle detection; it is realized using pixel profile analysis and RANSAC method to perform line fitting and extract the horizon;
\item Saliency detection module: as expressed in \cite{Achanta2009}, an image mask is built and the Euclidean distance between the color pixel vector in the gaussian blurred image and the average vector for the original color image is calculated; the detected salients will be given in the form of bounding boxes and are taken as the potential interested obstacles;
\item Harris corner extraction and tracking module: using the work proposed in \cite{Harris1988} and \cite{Bouguet1999}, motion evaluation is possible to distinguish surface obstacles from the potentials suggested by the saliency detection;
\item Obstacle detection module: the data coming from the previous modules are combined to generate the final results; since it is pratical to measure the dynamics of a potential object to verify its validity as an obstacle, a tracked feature with long lifespan in the salient bounding box is labelled as of high priority and this link the obstacles in consecutive frames.
\end{itemize}
\indent The stereo correspondence phase could be divided in three: an initial phase in which both cameras are calibrated so the results are used for stereo image undistortion and 3D reconstruction, an intermediate phase in which \textit{epipolar constraint} reduces the 2D search for obstacle correspondence, and then a \textit{stereo matching} phase in which the normalized cross correlation template matching method is adopted. Here, the bounding box of obstacled by monoculat obstacle detection in the left image is considered as the template window while the search is conducted along its epipolat line in the right image. In the end, a Kalman Filter is applied on the horizontal disparity in order to eliminate the stereo matching error and improve the range estimation accuracy.\\
\indent Another work similar to the previous one but that use only monular vision is the one proposed by Azzaby \textit{et al.} in \cite{Azzabi}. As before, the authors detect the horizon first and then the obstacles in the scene; last, they estimate the distance between the USV and objects, knowing the relative angle of each object and the horizion line. The algorithm developed for horizon detection can be summerized as follows:
\begin{itemize}
\item Grey level transformation: applied to the input image to reduce image time processing (despite of a little loss of information);
\item Sobel operator: it performs a 2D spatial gradient measurement on the image to emphasizes regions of high spatial frequency that correspond to edges;
\item Hough transform: a line passing through two pixels in the image domain must lie on the intersection of two lines in Hough domain;
\item Line election: it chooses the couple (\textit{a,b}) for which the line passes through the maximum of points and then traces this line that will be probably the horizon line.
\end{itemize}
\indent After this, the next step is detect objects in motion using optical flow estimation: given an input stream, the velocity of each pixel is firstly calculated; if a pixel has a velocity higher than a threshold value and is under the horizon, it is marked as moving and tracked and in the end the optical flow vector of each pixel is plotted in the frame.
\indent The output of the previous module is therefore used to estimate the distance from the USV to obstacles using a geometrical interpretation of the image. As result, the distance to the object, its width and its bearing angle is computed.

\section{Path Planner} \label{path_planner}
After having perceived the surrounding environment and the obstacles present in it with the techiques previously described, we have to realize a model of the world. This could be done in a two or three dimensional space but often the first option is preferred because it can guarantee less computational time to operate with it. Having a virtual representation of the world is a key point to plan a path for moving the autonomous vessel.\\
As described in \ref{introduction}, usually the \textit{Path Planner} module is usually divided in two sub-components: the \textit{global} one aims to find a safe path from the actual pose of the robot to a goal one, while the \textit{local} one tries to avoid moving obstacles close to the robot. This is a common way to develop a robust path planner but nothing prohibits  to model a unique one, able to plan a path and react to obstacles at the same time.\\
\indent In the following subsections I will describe the most recent path planners used in marine robotics, to guide autonomous vessels on the sea surface among other boats and moving hazards.

  \subsection{Global Path Planner} \label{gpp}
  The goal of the GPP is to continuosly modify the existing waypoint route to plan around obstacles detected with the long-range sensors. In the work of Larson \textit{et al.} \cite{Larson2007,Larson2007a}, the path planner use a two-dimensional (2D) obstacle map, a grid created by dividing the environment into a discrete grid and assigning each cell location a value representing the probability of being occupied or not by an obstacle. This is filled with stationary obstacles from the \textit{chart server} and moving obstacles provided by the radar. The underlying search technique is the A* search algorithm, chosen because it can find an optimal solution in a short amount of time. Since A* uses a cost analysis at each step, the author inserted an added cost for proximity to obstacles for allowing the USV to set a safety barrier around obstacles.\\
  To avoid moving obstacles tha path planner determines safe velocity ranges using the Velocity Obstacles method. This algorithm transform a moving obstacle into a stationary one by considering the relative velocity and trajectory of the USV wih respect to the obstacle, and in the end it returns a set of USV velocity vectors guaranteeing collision avoidance.\\
  In the case that changing the velocity doesn't avoid collision, the path planner changes path by creating a projected obstacle area (POA), for each obstacles and determining a safe alternative route using A* search. A POA is the area a moving obstacle could occupy in the future. This area is identified calculating the CPA: since a moving obstacle can pose a threat to the USV along multiple stretches of the path, it is necessary to calculate the CPA of every obstacle along each path segment.\\
  Among the previous concepts, the authors decided also to consider the International regulations for preventing Collisions at sea (known as COLREGS, for COLlision REGulationS). In this work the authors consider three primary COLREGs: crossing, overtaking and head-on situations. In the situation in which a traffic boat is crossing from the right, the vessel with the other on its starboard (right) side must give away. In the case the USV is overtaking a slow traffic boat, the USV must ensure enough clearance so that it keeps out of the way of the traffic boat being overtaken. Differenlty, if the USV ans the traffic boat are moving straight toward each other, both vessels must alter their course toward the starboard, so that they pass with the other vessel to its port(left side).\\
  The POA of a moving obstacle is calculated from the current path of the USV and the time taken to traverse that path. As the path changes, there is a need to update the POA and recalculate.\\

  \indent Casalino \textit{et al.} in \cite{Casalino2009} suggest an approach based on the \textit{visibility graph} concept and on a world-model in which obstacles are polygons and the robot is a point. A visibility graph is a graph of intervisible locations, typically for a set of points and obstacles in the Euclidean plane. Each node in the graph represents a point location, and each edge represents a visible connection between them. That is, if the line segment connecting two locations does not pass through any obstacle, an edge is drawn between them in the graph.\\
  In order to use the visibility Grph the obstacles insede the working are had to be transformed into polygons. At this pointthe Dikstra's Algorithm is applied between the starting point and the goal one can be applied and the resulting trajectory will not intersect any of the obstacles.

  \subsection{Local Path Planner} \label{lpp}
  A first example of LPP is given by the work of Kuwata \textit{et al.} \cite{Kuwata2014}, in which the authors suggest an algorithm to not only address hazard avoidance for stationary and moving hazards, but that also applies the COLREGS. In this work the authors consider three primary COLREGS: crossing, overtaking and head-on situations.\\
  Moreover, the proposed approach use the Velocity Obstacle (VO) concept: a velocity space \textit{v-$\theta$} grid (where \textit{v} denotes the USV speed and \textit{$\theta$} is the heading angle) is constructed as decision space to find the best velocity vector and the moving obstacle is expanded by the robot size. The reason to do this it to treat the robot as a point. As long as the robot's velocity lies outside the VO, it will not collide with obstacle, assuming that the velocity vectors are constant over time; if the velocity obstacle change over time, the VO-based approach reacts by replanning using the latest sesnsor information.\\
  The developed algorithm works in this way:
  \begin{itemize}
  \item Precollision Check: the closest point of approach (CPA) is computed with the current position and velocity of the USV and traffic vessels, and it is evaluated if any COLREGS rules need to be applied;
  \item Rule Selection: if the CPA meets temporal and distance conditions, the best COLREGS rule is applied analyzing a set of geometric constraints;
  \item Hysteresis: introduced to lower the rate at which the USV can change what rules to aply; this means that once a COLREGS maneuver is initiated, it continues to direct the boat for at least a minimun duration of time;
  \item Cost: once the constraints set of VO and COLREGS are generated, a defined cost for each \textit{$v_i$} and \textit{$\theta_j$} ammissible is generated and the (\textit{$v_i$}, \textit{$\theta_j$}) pair with the minimum cost is selected and the velocity command is sent to the vehicle controller.
  \end{itemize}

  \indent A similar approach has been implemented by Leng \textit{et al.}. In \cite{Leng2013} the authors integrate the Velocity Obstacle approach with Mixed Linear integer Programming (MILP). In solving the path planning prpblem, the dynamics and kinematics of the USV, sensors and uncertainty constraints of the environment all can be taken into consideration and linearized. In the end, the objective function will be
      \begin{center}
        min: $ \mid $ \textit{L$_{UG}$} - (\textit{v$_{UG}$} + \textit{$\Delta$ v$_{UG}$})$\Delta$t $ \mid$
      \end{center}
  where \textit{L$_{UG}$} and \textit{v$_{UG}$} denote the relative distance and velocity between the USV and the target point.\\
  As in \cite{Kuwata2014}, the collision is checked calculating the the CPA and its distance from the vessel. After this, six types of encounter situations are identified and the USV reacts depending on this decision.

  \indent Also the authors of \cite{Larson2007,Larson2007a} describe a local or \textit{reactive} path planner module; in their work, in addition of the global path planner, a local one is required because the long-range sensorsa are not capable of detecting small low-profile obstacles such as very small personal boats, o because the USV may inadvertently deviate from the planned path if GPS is jammed or the Inertial Navigation Unit (INU) drifts. \\
  In the adopted approach, all the near-field sensors are fused into a common level world-model, and individual behaviors vote on specific navigation solutions within that model. A number of arcs are projected in front of the vehicle over the local world-model obstacle map. The number of arcs considered is a function of the map size and grid spacing, with the arcs spaced such that one arc pases through each of the outer cells.This approach guarantees that each cell in the grid is covered by at least one arc so that all the navigable path are considered. Each arc is given a weight or vote based on the distance the robot could travel along the arc before it encountered an obstacle. The votes are scalded from 0 to -1 so that they can be combined with votes from oher navigation behaviors.\\

  \indent In \cite{Casalino2009} is described a different approach for a reactive path planner based on the \textit{bounding box} concept. A bounding box of a track is easily defined as the rectangle that defines the area the USV should avoid in order to guarantee a safety distance from the moving object. The algorithm proposed suggest to integrate the USV actual position S and the local goal G to a graph made of the four edges of the box, by adding the edges starting from either S or G and ending on one of the vertexes of the box that are not intersecting any edge. Then the solution would be any path from S to G, which can be obtained by any graph taversing algorithm, such as A* or Dijkstra's algorithm. The problem with this approach is its sub-optimality because it does not take into account any kinematic property of the tracks nor the USV.\\
  The algorithm implemented to reach the locally optimal avoidance is composed as follows, using an A* search:
  \begin{itemize}
        \item For each visible vertex of the bounding box, compute the angle $\theta$ such that the vessel can intercept it;
        \item Calculate time(cost) \textit{$\bar{t}$} and position $\bar{P}$ of intercept;
        \item Calculate estimated time to goal \textit{h} $\equiv$ \textit{G} - \textit{$\bar{P}$} / \textit{$v_1$}, where \textit{$v_1$} is vehicle's speed;
        \item Compute global estimate cost \textit{f} = \textit{$\bar{t}$} + \textit{h}.
  \end{itemize}
  At each iteration the node with the lowest \textit{f} is selected. At each iteration pop the node with lowest \textit{f} from the \textit{openset}, the set of all the nodes that have been created but not yet explored in the search, check whether the goal can be reached without collision, if so end the search. Otherwise, for each obstacle, calculate the vertexes intercept positions and check if this path is collision free with \textit{ray tracing} techinque. If so, add this node to the openset and proceed back to the first step.\\
  \indent In the successive work \cite{Simetti2014} the authors present a refinement of this algorithm. To avoid crossing the bounding box and thus failing the obstacle avoidance, they introduce a \textit{safety bounding box} around the original collision bounding box. All the computations are now performed against it, and its vertexes are used to performed the avoidance. If for same reasons the USV enters the safety box, it must exit it without crossing the main diagonals, ensuring that the vehicle moves away from the collision bounding box.\\
  To increase the likelihood that the path is still collision free with the safety bounding box even under estimation uncertainty, a \textit{supporting bounding box} is adopted, whose dimensiond depends on the position of the USV:
  \begin{itemize}
        \item if the USV is inside the safety bounding box, the supporting one coincides with teh safety one;
        \item if the USV is far away with the safety bounding box, the supporting one coincided with a maximum bounding box;
        \item if the USV is in-between the maximum bounding box and the safety one, the supporting one varies with the distance, shrinking as the USV is closert ot the safety one.
  \end{itemize}
  In this way, when the USV if far away from the incoming vessel, the computed path will be very robust to changes in speed and heading of the obstacle and its safety bounding box.\\

  A different approach based on lane-constrained trajectory genration is proposed in \cite{Tan2010}. In the author's work, while avoiding obstacles, the vessel has to meet some objectives. Firstly, it is required to maintain a minimum distance from each obstacles at all times (\textit{safety distance} objective). Secondly, the USV must observe the COLREGs rules. The last two objectives are called \textit{cross track} and \textit{shortest time} objectives: they means the USV should keep as close as possible to the intended path and complete it the the shortest path possible.\\
  The idea presented is divided in two step:
  \begin{itemize}
        \item Maneuver generation: the platform's motion is forward simulated for a fixed number of time steps using simple models of the maneuvers tracker and the boat;
        \item Maneuver selection: is a multi-stage process in which objectives, divided in \textit{rules} and \textit{criteria}, are ordered based on fixed priorities; at each stage, a single objective is considered and candidate are eliminated based on that particular objective. As long as more than one candidate emrges from the elimination, the remaining candidates are subjected to the next stage of elimination.
  \end{itemize}
  A limitation of this approach is given by the generation step: because only a sample of possible maneuvers is taken, the algorithm may sometimes be unable to find a solution.

  \indent In \cite{Blaich2015} it's proposed a specialized A* algorithm that allows velocity variations and considers different turning circles for different velocities. First of all. a three-dimensional occupancy grid is built by inserting the measurementd of the laser finder which are transformed to a local reference frame. Then the map is processed and contours of obstacles are extracted.\\
  In parallel to the mapping procedure for static obstacles, a Multi Object Tracket (MOT) is implemented for moving objects.\\
  For the evasive path generation, an A* algorithm is specialised such that it considers static obstacles, tracked agents and the mission path. A penalty depending on the total lenght of the path that had to be skipped is added to the cost function of the A* algorithm. Thir results in evasive path that lead back to the mission path after avoiding an obstacle. However, to guarantee the feasibility of an evasive path, the kinematic constraints of the USV have to be considered during the path serach. To use the full dynamic capability of the vessel, it is necessary to allow changing velocities in the evasive path along with the respective changes of the minimum turning circle. These alternations in velocity are enabled by adding both the velocity and the time to the search space, that consist now of four dimensions - two for position and one dimension for each time and velocity.

  Tang \textit{et al.} in \cite{Tang2012} shows a new method called Obstalce Avoidance Algorithm Based on Heading Window (OAABHW) for the near-field static obstacle avoidance of USVs. The algorithm trasnforms dynamic windows into heading window and translational velocity window by using \"Divide and Conquer\" Strategy, and grasp an optimized avoidance angle from solving the constraint optimization problem of heading window at second step. Then the rotational velocity could be determined by the navigation angle and heading of USV. The translational velocity of avoidance is produced by the translational velodity model which is formed by the rotational velocity and the distribution of obstacles in the surroundings.\\
  Based on the concept that the translational velocity is less important than the heading angle, the authors build the relative coordinate of USV and upload all the obstacles to it. Then the constraint set of heading angle of USV can be calculated by dealing the obstacle in near-field on tangent method. Obstacles are replaced by their circumcircle, and the geometry size of teh USV is regarded as the expanded factor of obstacle expanding, meanwhile USV can be trated as a particle.\\
  In the near-field of USV, a number of virtual rays projected around the USV with a certain angular resolution, any angle obstructed by obstacles is defined as infeasible heading angle of USV. Upon the analysi of obstalces in the relative coordinate of USV by tangent method, the maximum angle and minum angle could be grasped and recored as $\theta_{obs\_max}$ and $\theta_{obs\_min}$. Then these two angles are transferred from relative coordinate to absolute ones.\\
  In the process of excellent heading angle selection, the degree of heading yaw is treated as optimization objective. the heading window and the infeasible set of heading angle are treated as constraints. By solving the single optimization problem, the best navigation angle can be grasped. The constrainets optimzation problem can be mathematically formulated as:
      \begin{center}
            $max: F_{Head} (\theta) = 1 - \frac{\theta_{goal} - \theta}{2\pi}$
      \end{center}
  where $\theta_{goal}$ is the basic value of poptimal angle selection which is formed by global target point and the current position of USV.\\
  The excellent avoidance navigation angle $\theta_{out}$ is obtained and the avoidance rotational velocity $\omega_{out}$ could be calculated by it and the current heading angle $\theta_{USV}$:
      \begin{center}
            $\omega_{out} = \omega_c + \frac{2(\theta_{out} - \theta_{USV} -\omega_c * \Delta t)}{\Delta t}$
      \end{center}
  where $\omega_c$ is the current rotational velocity of USV, and $\Delta t$ is time\-window.\\
  The translational velocity strongly depends on the rotational one and on the obstacles distributions. If obstacles are close to USV or the avoidance rotational velocity $\omega_{out}$ is comparatively large, then it has to slow the translational velocity to pass the hazard areas. If obstacles are far away from USV and rotational velocity is small, then USV should navigate at high speed.
\section{Conclusion} \label{conclusion}



% Can use something like this to put references on a page
% by themselves when using endfloat and the captionsoff option.
\ifCLASSOPTIONcaptionsoff
  \newpage
\fi

% references section

% can use a bibliography generated by BibTeX as a .bbl file
% BibTeX documentation can be easily obtained at:
% http://www.ctan.org/tex-archive/biblio/bibtex/contrib/doc/
% The IEEEtran BibTeX style support page is at:
% http://www.michaelshell.org/tex/ieeetran/bibtex/
%\bibliographystyle{IEEEtran}
% argument is your BibTeX string definitions and bibliography database(s)
%\bibliography{IEEEabrv,../bib/paper}
%
% <OR> manually copy in the resultant .bbl file
% set second argument of \begin to the number of references
% (used to reserve space for the reference number labels box)

\bibliography{bibliography}
\bibliographystyle{plain}




% that's all folks
\end{document}
